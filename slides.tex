\documentclass{beamer}
\usepackage{graphicx}
\usepackage{caption}
\usepackage{latexsym} 
\usepackage{cmap}
\usepackage[russian, english]{babel}
\usepackage[T2A]{fontenc}
\usepackage[utf8]{inputenc}
\usepackage{natbib}
\usepackage{amsmath}
\usetheme{Warsaw}
\usecolortheme{whale}
\begin{document}
\setbeamertemplate{navigation symbols}{}%remove navigation symbols

\title[Сравнение алгоритмов классификации]{Сравнение алгоритмов классификации на предмет устойчивости к зашумлению входных данных}
\author[Владимир Назаренко]{Владимир Назаренко\\{\scriptsize Научный руководитель: \\ \vspace{-2mm} д.ф.-м.н., проф. Б.А. Новиков}}
\institute{Санкт-Петербургский Государственный Университет\\
Математико-механический факультет}
\date{Санкт-Петербург\\ 2014}
\subject{Алгоритмы классификации}

\large
\frame{\titlepage}
\setbeamerfont{page number in head/foot}{size=\normalsize}
\setbeamertemplate{footline}[frame number]

\begin{frame}
    \frametitle{Обзор предметной области}
    %Здесь про машинное обучение в целом
    \begin{itemize}
    	\item{Машинное обучение находит своё применение в большом количестве 				задач}
    	\item{Обучение с учителем}
    	\item{Классификация}
    	\item{Данные часто ошибочны}
	   	\item{Некорректные данные влекут увеличение количества ошибок 				классификации}
    	\item{Средство анализа данных Weka}
    \end{itemize}
  \end{frame}
  
  \begin{frame}
    \frametitle{Постановка задачи}
    \begin{itemize}

    	\item{Выполнить сравнение качества работы некоторых алгоритмов классификации при наличии шумов в тестовом множестве, где мерой качества является количество верно классифицированных элементов в тестовом множестве}
    	
    	\item{Провести анализ изменений качества работы алгоритмов при увеличении интенсивности шума и провести анализ причин изменений качества работы}
    	
    \end{itemize}
  \end{frame}
  
  \begin{frame}
    \frametitle{Работы в данной области}
    %Здесь про машинное обучение в целом
    \begin{itemize}
    	\item{Ассиметричное зашумление аттрибутов(Mannino et al. 2009)}
    	\item{Деградация алгоритма "наивный байес" на зашумленных данных(Glick et al. 2004)}
    \end{itemize}
  \end{frame}
  
  
  
  \begin{frame}
  \frametitle{Описание классификаторов (1)}
    \framesubtitle{Классификатор на основе дерева выбора}
    \begin{itemize}
    	\item{Дерево выбора}
    	\item{Применение дерева выбора в классификации}
    	\item{Проблема выбора разделения}
    	\item{Random Tree}
    \end{itemize}
  \end{frame}
  
  \begin{frame}
  \frametitle{Описание классификаторов (2)}
    \framesubtitle{Наивный байесовский классификатор}
    \begin{itemize}
    	\item{Теорема Байеса}
    		$P(y=C|x)=\frac{P(C) P(x|y=C)}{P(x)}$
    	\item{Следствия независимости свойств объекта\\}
    	$$P(y=C|x)=\frac{P(C) \prod_{i=1}^n{P(x_i|y=C)}}{\prod_{i=1}^n{P(x_i)}}$$
    	$$\max_{C \in \{C_1,..,C_n\}}P(y=C|x)$$
    	\item{Naive Bayes}
    \end{itemize}
  \end{frame}
  
  \begin{frame}
  \frametitle{Описание классификаторов (3)}
    \framesubtitle{Классификатор на основе метода опорных векторов}
    \begin{itemize}
    	\item{Объект представляется точкой}
    	\item{Нужно найти лучшую разделяющую плоскость}
    	\item{Проблема максимального запаса}
    	\item{SMO}
    \end{itemize}
    \begin{figure}
    \centering
    \includegraphics[scale=0.15]{graphics/SVM_margins}
  \end{figure}
  \end{frame}
  
  
  
  \begin{frame}
    \frametitle{Описание эксперимента}
    	\begin{itemize}
    	\item{В качестве набора данных был выбран MNIST Database}
    	\item{На вектор свойств наносился шум случайным образом}
    	\begin{figure}[ht!b]
		\centering
		\begin{minipage}{.25\textwidth}
			\includegraphics[width=\textwidth]{graphics/digits-nr0.jpg}
			\captionsetup{justification=centering}
			\caption*{Изображения без\\ шума}
		\end{minipage}
		\begin{minipage}{.25\textwidth}
			\includegraphics[width=\textwidth]{graphics/digits-nr30.jpg}
			\captionsetup{justification=centering}
			\caption*{ 30\% пикселей\\ испорчены}
		\end{minipage}
		\begin{minipage}{.25\textwidth}
			\includegraphics[width=\textwidth]{graphics/digits-nr60.jpg}
			\captionsetup{justification=centering}
			\caption*{ 60\% пикселей\\ испорчены}
		\end{minipage}
		\label{fig:noise}
		\end{figure}
		\item{Выбранные алгоритмы тестировались в двух ситуациях}
    		\begin{itemize}
    		\item{Зашумлено только тестовое множество}
    		\item{И тестовое, и тренировочное множество зашумлены}
    		\end{itemize}
    \end{itemize}
  \end{frame}
  
  
  \begin{frame}
  	\frametitle{Технические детали}
  		\begin{itemize}
  		\item{Для нанесения шумов и конвертации набора данных использовался скрипт 					на Python}
    	\item{Все параметры алгоритмов были оставлены стандартными, кроме 					параметра complexity для SMO}
		\item{Для SMO параметр complexity был подобран}
		\begin{figure}[ht!]
		\includegraphics[width=\textwidth]{graphics/c_param.pdf}
		\captionsetup{justification=centering}
		\label{fig:SMO_qual}
		\end{figure}
    \end{itemize}
  \end{frame}
  
  \begin{frame}
    \frametitle{Результаты эксперимента(1)}
    \begin{figure}[ht!]
	\includegraphics[width=\textwidth]{graphics/Perf_noised.pdf}
	\captionsetup{justification=centering}
	\caption{Эффективность алгоритмов при построении модели на зашумленных 			данных}
	\label{fig:perf1}
	\end{figure}
  \end{frame}
  
  \begin{frame}
    \frametitle{Результаты эксперимента(2)}
    \begin{figure}[ht!]
	\includegraphics[width=\textwidth]{graphics/Perf_unnoised.pdf}
	\captionsetup{justification=centering}
	\caption{Эффективность алгоритмов при построении модели на незашумленных 			данных}
	\label{fig:perf2}
	\end{figure}
  \end{frame}
  
  \begin{frame}
    \frametitle{Выводы}
    \begin{itemize}
	\item{Алгоритм ``Random Tree'' оказался неустойчив к шуму}
	\item{Алгоритм ``Naive Bayes'' показал удовлетворительные результаты при обучении на зашумленном множестве и неудовлетворительные при обучении на незашумленном}
	\item{Классификатор ``SMO'' показал лучшие результаты из представленных алгоритмов}
	\end{itemize}
  \end{frame}
  
  \begin{frame}
    \frametitle{Результаты}
    \begin{itemize}

		\item{Выполнено сравнение трёх популярных алгоритмов классификации на 				предмет устойчивости к шумам в тестовом множестве}
		
    	\item{Проведён анализ причин устойчивости/неустойчивости алгоритмов к 					шумам}
    	
    	\item{Построено тестовое окружение, позволяющее легко воспроизвести 				эксперимент}
    	
    \end{itemize}
  \end{frame}
  
\end{document}
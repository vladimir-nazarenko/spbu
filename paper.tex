\documentclass{article}
\usepackage[russian]{babel}
\usepackage[utf8]{inputenc}
\begin{document}
\section{Постановка задачи}
Изобилие алгоритмов машинного обучения вынуждает сравнивать их между собой. К сожалению данные, предоставленные для обработки зачастую содержат ошибки и неточности. Не секрет, что критериев такого сравнения можно придумать очень много, так почему бы не сравнить алгоритмы по устойчивости к порче входных данных? Для этого рассмотрим три популярных алгоритма: SVM, Decision Tree и Naive Bayes и рассмотрим их на датасете рукописных цифр, которые будем постепенно портить.
\end{document}
\documentclass{article}
\usepackage{cmap}
\usepackage[russian]{babel}
\usepackage[utf8]{inputenc}
\begin{document}
\section{Постановка задачи}
Изобилие алгоритмов машинного обучения вынуждает сравнивать их между собой. К сожалению данные, предоставленные для обработки зачастую содержат ошибки и неточности. Не секрет, что критериев такого сравнения можно придумать очень много, так почему бы не сравнить алгоритмы по устойчивости к порче входных данных? Для этого рассмотрим три популярных алгоритма: SVM, Decision Tree и Naive Bayes и рассмотрим их на датасете рукописных цифр, которые будем постепенно портить.
\section{Эксперименты}
Итак, в качестве датасета был выбран MNIST Database, так как изображения в нём подвегнуты качественному перпроцессингу и мы можем быть уверены, что изображение испортили именно мы. Портить изображение можно несколькими способами, например, уже испробовано засвечивание некоторого процента пикселей изображения. Также планируется трансформировать изображение более естественными путями, такими как поворот или растяжение. Затем нужно с помощью пакета анализа данных, в частности, weka, проверить выбранные алгоритмы следующими способами:
\begin{itemize}
\item Запустить построение гипотезы на неиспорченных данных, а проверять гипотезу на испорченных данных
\item Строить гипотезу и проверять её на испорченных случайным образом данных(например, шум). 
\end{itemize}
\end{document}